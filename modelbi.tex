\documentclass[a4paper,12pt]{amsart}

\usepackage[usenames,dvipsnames]{xcolor}


\usepackage{
   amssymb
  ,amsfonts
  ,amsmath
  ,tikz
  ,xspace
  ,lmodern
  ,datetime
  ,mathtools}


\usepackage[utf8]{inputenc}
\usepackage[T1]{fontenc}




% %
% %




\usepackage[cal=boondoxo]{mathalfa}


\def\proP{\textsc{p}}
\def\proQ{\textsc{q}}
\def\proH{\textsc{h}}
\def\proL{\textsc{l}}
\def\proK{\textsc{k}}
\def\proG{\textsc{g}}

\providecommand{\abbrv}[1]{#1.\@\xspace}
  \providecommand{\ie}{\abbrv{i.e}}
  \providecommand{\etc}{\abbrv{etc}}
  \providecommand{\prof}{\abbrv{prof}}
  \providecommand{\viz}{\abbrv{viz}}
  \providecommand{\eg}{\abbrv{e.g}}
  \providecommand{\achap}{\abbrv{Ch}}
  \providecommand{\adef}{\abbrv{Def}}
  \providecommand{\aprop}{\abbrv{Prop}}
  \providecommand{\athm}{\abbrv{Thm}}






\usepackage{amsthm}

\newtheoremstyle{reference}
   {}                
   {}                
   {}              
   {}                      
   {\fontseries{b}\selectfont}              
   {:}                     
   {.2em}                  
   {\thmname{#1}           
    \thmnumber{#2}         
    \thmnote{{\sc [#3]}}} 

\theoremstyle{reference}
  \newtheorem{theorem}{Theorem}[section]
  \newtheorem{lemma}[theorem]{Lemma}
  \newtheorem{proposition}[theorem]{Proposition}
  \newtheorem{example}[theorem]{Example}
  \newtheorem{exercise}[theorem]{Exercise}
  \newtheorem{remark}[theorem]{Remark}
  \newtheorem{definition}[theorem]{Definition}
  \newtheorem{corollary}[theorem]{Corollary}
  \newtheorem{notat}[theorem]{Notation}
  \newtheorem*{acknowledgements}{Acknowledgements}
  \newtheorem{scholium}[theorem]{Scholium}
  \newtheorem{counterex}[theorem]{Counterexample}
  \newtheorem*{theorem*}{Theorem}
  \newtheorem*{lemma*}{Lemma}
  \newtheorem*{proposition*}{Proposition}
  \newtheorem*{example*}{Example}
  \newtheorem*{exercise*}{Exercise}
  \newtheorem*{remark*}{Remark}
  \newtheorem*{definition*}{Definition}
  \newtheorem*{corollary*}{Corollary}
  \newtheorem*{notat*}{Notation}
  \newtheorem*{scholium*}{Scholium}
  \newtheorem*{counterex*}{Counterexample}


\newcommand{\opp}{\text{op}}


\newcommand{\cate}[1]{\text{\fontseries{b}\selectfont{#1}}}
  \newcommand{\A}     {\cate{A}}
  \newcommand{\B}     {\cate{B}}
  \newcommand{\C}     {\cate{C}}
  \newcommand{\Cat}   {\cate{Cat}}
  \newcommand{\D}     {\cate{D}}
  \newcommand{\K}     {\cate{K}}
  \renewcommand{\P}   {\cate{P}}
  \newcommand{\catS}  {\cate{S}}
  \newcommand{\Sets}  {\cate{Sets}}
  \newcommand{\sSet}  {\cate{sSet}}
  \newcommand{\Top}   {\cate{Top}}

  \newcommand{\E}     {\mathcal{E}}
  \renewcommand{\L}   {\mathcal{L}}
  \newcommand{\M}     {\mathcal{M}}
  \newcommand{\R}     {\mathcal{R}}
  \newcommand{\V}     {\mathcal{V}}
  \newcommand{\F}     {\mathfrak{F}}
  \newcommand{\T}     {\mathfrak{T}}

\newcommand{\pto}{\leadsto}
\newcommand{\wk}{\textsc{wk}}
\newcommand{\cof}{\textsc{cof}}
\newcommand{\fib}{\textsc{fib}}










% %%%%%%%%%%%%%%%%%%












% \usepackage{lmodern-standard}

\def\proP{\textsc{p}}
\def\proQ{\textsc{q}}
\def\proH{\textsc{h}}
\def\proL{\textsc{l}}
\def\proK{\textsc{k}}
\def\proG{\textsc{g}}

\title{locally model bicategories}
\author{FL BC}
\begin{document}
\begin{abstract}
The theory of bicategories $\A$ where each hom-category $\A(X,Y)$ has a model structure.
\end{abstract}
\maketitle
\def\Prof{\cate{Prof}}
\section*{Introduction}
Here be intro

We recall some of the definition which will be useful later, to mantain a certain degree of self-containedness.
\begin{definition}[Two-variable adjunctions]

\end{definition}
\begin{definition}[Monoidal model category]
A symmetric monoidal model category $\C$ is a model category equipped with the structure of a closed symmetric monoidal category, such that these two structure interact well:
\begin{itemize}
\item 
\item 
\end{itemize}
\end{definition}
\begin{definition}[Left and Right Quillen functor]

\end{definition}
\begin{definition}[Quillen two-variable adjunction]

\end{definition}

\section{The model bicategory of profunctors}
\def\Prof{\cate{Prof}}
\begin{definition}[The bicategory of $\V$-profunctors]\label{profdef}
Let $\V$ be a Bénabou cosmos\footnote{\ie a symmetric monoidal closed, bicomplete category}. Define a bicategory $\Prof$ having
\begin{itemize}
\item 0-cells (objects) those of $\V$-$\Cat$ (small $\V$-categories $\A,\B,\C,\D,\dots$);
\item 1-cells $\proP,\proQ,\dots$, depicted as arrows $\A\pto \B$, the functors $\A^\opp\times\B\to \V$;
\item 2-cells $\alpha\colon\proP\Rightarrow\proQ$ the natural transformations between these functors.
\end{itemize}
Given two contiguous 1-cells $\A\stackrel{\proP}{\pto}\B\stackrel{\proQ}{\pto}\C$ we define their \emph{composition} $\proQ \diamond \proP$ as the coend
\[
\proQ \diamond \proP(a,c) := \int^x \proP(a,x)\times\proQ(x,c)
\]
\end{definition}
The 1-cells of $\Prof$ are called \emph{profunctors}, or more rarely \emph{distributors} (following the equation $\text{funct\emph{ions}} : \text{funct\emph{ors}} = \text{distribut\emph{ions}} : \text{distribut\emph{ors}}$), \emph{correspondences} (consider the case of $\mathcal V=\{0,1\}$ where $\A, \B$ are discrete categories), or \emph{bimodules} (consider the case where $\mathcal V=\cate{Ab}$ and $\A,\B$ are rings; see \cite{nashphd} for applications in this sense).
\begin{remark}
There is an alternative, but equivalent definition for $\proQ \diamond \proP$ which exploits the universal property of $\widehat{\C}$: any profunctor $\proP\colon \A\pto \cate B$ can be identified with its \emph{mate} under the adjunction giving the cartesian closed structure of $\Cat$,
\[
\Fun(\A^\opp\times \B ,\Sets)\cong \Fun(\B, [\A^\opp,\Sets])
\]
\ie with a functor $\widehat\proP\colon\cate B\to \widehat{\A}$ obtained as $b\mapsto \proP(\firstblank,b)$. Hence we can define the composition $\A\stackrel{\proP}{\pto}\cate B\stackrel{\proQ}{\pto}\C$ to be $\Lan_\yon \widehat\proP\circ\widehat\proQ$:
\[\xymatrix{
& \B \ar[d]_{\yon }\ar[r]^{\widehat\proP}& \widehat{\A} \\
\C \ar[r]_{\widehat\proQ}& \widehat{\B} \ar[ur]_{\Lan_\yon \widehat\proP}& 
}\]
\end{remark}
This is equivalent to the previous definition, in view of the characterization
of a left Kan extension as a coend in $\widehat{\A}$ (\cite[Eq.
(37)]{cofriend}):
\[
\Lan_\yon \widehat\proP\cong \int^b\widehat{\B}(\yon_b,\firstblank)\cdot \widehat\proP(b).
\]
Since in $\Sets$ copower coincides with product (\ie $X\cdot Y\cong X\times Y$,
since $\Sets(X\cdot Y,B)\cong \Sets(X,\Sets(Y,B))\cong \Sets(X\times Y,B)$,
naturally in $B$), we have
\begin{align*}
\Lan_\yon \widehat{\proP}(\widehat{\proQ}(c)) &\cong \int^b
	\widehat{\B}(\yon_b,\widehat{\proQ}(c))\cdot \widehat{\proP}(b)\\
&\cong \int^b \widehat{\proQ}(c)(b)\cdot \widehat{\proP}(b)\cong \int^B \proP(\firstblank,b)\times \proQ(b,c).
\end{align*}
\begin{lemma}
Composition of profunctors is a bifunctor 
\[
\diamond \colon \Prof(\A,\B)\otimes \Prof(\B,\C)\to \Prof(\A,\C).
\]
\end{lemma}
\begin{proof}
The only nontrivial thing to verify is the ``interchange'' law
\[
nohdfoainfoa
\]
This is easy in view of the fact that the diagram
\[
fnaoinfoa
\]
commutes, due to the bifunctor $\otimes \colon \V\times \V \to \V$, and then
co-ending this commutative square over $b\in\B$ preserves this property giving
the result.
\end{proof}
\begin{example}
The canonical example of a \emph{locally model 2-category} is the bicategory
$\V$-profunctors where $\V$ is a \emph{model} cosmos, \ie
\begin{itemize}
\item $\V$ is still a symmetric monoidal closed, bicomplete category,
\item with a model structure $(\wk,\cof,\fib)$
\item such that the ``monoid axiom'' is fulfilled.
\end{itemize}
Each $\Prof(\A,\B)$ can be endowed with many model structures (injective or
projective, defining classes objectwise on $\V$-natural transformations
$\eta\colon F\Rightarrow G$ between $\V$-functors).
\end{example}
Composition of profunctors is defined so and so.
\begin{remark}
Composition of profunctors satisfies the interchange rule.
\end{remark}
\begin{remark}
Composition of profunctors has adjoints on such and such sides.
\end{remark}
\begin{definition}
The injective-projective model structure.
\end{definition}
\begin{proposition}
The composition functor is a left Quillen; this entails in particular that the pushout product $\hat\diamond$ of two cofibrations is again a cofibration, and spushout product $\hat\diamond$ of two cofibrations is again a cofibration, and some other things.
\end{proposition}
\section{The abstract theory of model bicategories}
Other examples
\begin{itemize}
\item monoidal categories and lax profunctors, operads and "operadic profunctors" etc.
\item take a simplicial category $C$ and contruct a locally model 2-category $\widetilde{C}$ whose objects are the same as those of $C$ and whose mapping model categories are the slices $\sSet/[X,Y]$. 
\item If $\V$ is cartesian closed then you can probably do the same with a generic $\V$-enriched category.
\end{itemize}
A general theory

\bibliography{allofthem}{}
\bibliographystyle{amsalpha}
\hrulefill
\end{document}
